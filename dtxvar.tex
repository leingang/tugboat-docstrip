\documentclass{ltugboat}
%%% Start of metadata %%%
\AtBeginDocument{
    \title{Using \docstrip{} for Multiple Document Variants}
}
\author{Matthew Leingang}
\address{New York University \\ 251 Mercer Street \\ New York, NY 10012\\ USA}
\netaddress{leingang@nyu.edu}
\personalURL{www.cims.nyu.edu/~leingang/}
%%% End of metadata %%%

\usepackage{microtype}
\usepackage{graphicx}

\usepackage{listings}
\lstset{
    columns=flexible,
    keepspaces=true,
    commentstyle=\slshape,
    language=[LaTeX]TeX,
    frame=tb,
    numbers=left,
    basicstyle=\small\ttfamily}


\newcommand{\cls}[1]{\texttt{#1}} % class name
\newcommand{\pkg}[1]{\texttt{#1}} % package name
\renewcommand{\env}[1]{\texttt{#1}}% environment name (no \begin as in original)
\newcommand{\fn}[1]{\texttt{#1}}  % file name
\newcommand{\opt}[1]{\texttt{#1}} % DocStrip option
\newcommand{\prog}[1]{\textit{#1}} % program

\usepackage{ifpdf}
\ifpdf
  \usepackage[breaklinks,hidelinks,pdfa]{hyperref}
\else
  \usepackage{url}
\fi
\newcommand{\docstrip}{\texorpdfstring{\textsf{DocStrip}}{DocStrip}}

\begin{document}

\maketitle

\begin{abstract}
    I describe a method of keeping multiple variants of the same document
    within the same file using \docstrip.
\end{abstract}

\section{Introduction}

As a college professor, there are several times when I need to keep teaching
materials in several different forms.  For example:

\begin{itemize}
    \item A single class day's lesson might consist of lecture slides in the
    \cls{beamer} class, a handout of the same slides printed 2–3 on a page for
    student notes, my own lecture notes in \cls{beamer} with the
    \pkg{beamerarticle} package loaded, a worksheet for distribution, and
    solutions to the worksheet.

    \item A single week's homework assignment might consist of problem
    statements, with hints and reading notes, a \LaTeX{} template for students
    to fill in with their own answers, and solutions with comments to be
    published after the assignment has been graded.
\end{itemize}

Over the years I have developed a workflow for writing these ``bundles'' of
documents in the same file, using the \LaTeX{} \docstrip{} utility.  In this
article I will introduce the reader to \docstrip{} and explain how I use it.

\section{The \docstrip{} utility}

\docstrip{} \cite{docstrip} was originally designed as a literate programming method.  \LaTeX{}
package and class authors use it to write documentation for their code in the same
file as the implementation, within commnted lines.  \docstrip{} would
\emph{strip} out the comment lines and use them to produce \emph{doc}umentation.
The slimmer package and class files would be installed to save compile time.  In
subsequent versions \docstrip{} developed the ability to write lines to several
different files in one batch, through the use of options.

\begin{lstlisting}[float,
    numbers=none,
    caption={A minimal \docstrip{} batch file},
    label={lst-minbatch}
]
    \input docstrip.tex 
    \generate{\file{foo.sty}
             {\from{foo.dtx}{package}}}
    \endbatchfile
\end{lstlisting}

\docstrip{} batches are programmed in a \TeX{} file as in
Listing~\ref{lst-minbatch}.  The \cs{input} line loads the \docstrip{} code.
The \cs{generate} line instructs, effectively, ``Read \fn{foo.dtx} and write
\fn{foo.sty}, setting the \opt{package} option.''

When generating files, \docstrip{} ignores all lines beginning only with a
single \verb!%!.  Non-commented lines are written to all destination files.
Lines beginning with a \emph{guard} will be written to the destination file
depending on the options set.  Each guard begins with a \verb!%! and contains a
boolean expression enclosed by angle brackets.  For example, a line beginning
with \verb|%<bar>| will only be passed to generated files when the \opt{bar}
option is set.

Now putting guards at the start of every line would be cumbersome to type.  So
guard modifiers are used to delimit blocks of code with the same guard.  Any
expression preceded by `\verb|*|' will apply the indicated guard to every line
that follows, until the identical expression is encountered with the `\verb|/|'
modifier.  This gives an almost \acro{HTML}-like layer to the \docstrip{} source
file, where blocks of code between lines starting with \verb!%<*bar>! and
\verb!%</bar>! will be written to any file generated with the \opt{bar} option.

Remarkably, a \docstrip{} batch declaration can be put \emph{at the beginning}
of a \docstrip{} source file.  Running \TeX{} on the file will instruct \TeX{}
to parse the file a second time, this time writing the individual destination
files.  In this way, \docstrip{} files can be configured to ``self-extract.''

\section{Example: A problem set with answer template and solutions}
\label{sec-exam}

As a small but not quite minimal example, let's consider a \docstrip{} file
called \fn{hw.dtx}.  The entire file can be viewed online as part of the github
repository for this article: \url{github.com/leingang/tugboat-dtxvar}.

\subsection{Batch header}

Listing~\ref{lst-dtx-hdr} shows the beginning of \fn{hw.dtx}, which loads
\fn{docstrip.tex} and declares the \cs{generate} batch.  It's surrounded with
\verb|driver| guards.  Since no generated file sets the \verb|driver| option,
this block is not written to \emph{any} file.
\lstinputlisting[
    firstline=1,lastline=11,
    float=*,
    label={lst-dtx-hdr},
    caption={The header block of \fn{hw.dtx}
    declaring \cs{generate} batch}
]{hw.dtx}

We generate four files:

\begin{itemize}
    \item \fn{hw.qns.tex}, which sets the \opt{questions} option.  This document
    will be the prepared questions sheet for the instructor to distribute to the
    students.
    \item \fn{hw.ans.tex}, which sets the \opt{questions} and \opt{answers}
    options.  This is intended to be distributed to students as a template to
    fill in their answers.
    \item \fn{hw.sol.tex}, which sets the \opt{questions} and \opt{solutions}
    options.  This document can be published once the assignment is collected
    and graded.
    \item \fn{hw.bib}, which only sets the \opt{bib} option.  This is a
    \BibTeX{} file that can be included in any of the \LaTeX{} files.
\end{itemize}

Lines 12–89 of \fn{hw.dtx} are delimited with \verb|<questions>| guards and
enclose an entire \LaTeX{} document from \verb|\documentclass{article}| to
\verb|\end{document}|.  Lines 90 and onward are delimited with \verb|<bib>| and
comprise the complete \fn{hw.bib} file.

\subsection{Using guards to conditionally include text}

Listing~\ref{lst-dtx-qna} shows an excerpt of \fn{hw.dtx} that declares a
question.   Notice that the \env{hint} environment is surrounded by a guard with
a compound boolean expression \verb|<!answers&!solutions>|.  The effect is that
the  \env{hint} is shown when the \opt{questions} option is selected, but
\opt{answers} and \opt{solutions} are \emph{not} selected.  That is, only in the
\fn{hw.qns.tex} file.  The resulting block that is written to the questions file
is shown in Listing~\ref{lst-tex-qns}.

\lstinputlisting[
    firstline=61,lastline=83,firstnumber=61,
    float=*,label={lst-dtx-qna},
    caption={An excerpt of \fn{hw.dtx} declaring a question (the question is from
        \cite{Scheinerman})}
]{hw.dtx}
\lstinputlisting[
    firstline=61,lastline=67,firstnumber=61,
    float=*,label={lst-tex-qns},caption={The generated block in \fn{hw.qns.tex}}
]{hw.qns.tex}

Comment lines that begin with a single \verb|%| are stripped from the input and
do not get printed to any output file.  But comment lines beginning with
\emph{two} \verb|%| characters remain.  So the comment on line 71 of \fn{hw.dtx}
is retained in the \fn{hw.ans.tex} file (Listing~\ref{lst-tex-ans}).  It is a
note to the student where to write their answer.  Finally, the \env{solution}
environment and subsequent commentary paragraph are written to the
\fn{hw.sol.tex} file (Listing~\ref{lst-tex-sol}).

\lstinputlisting[
    firstline=58,lastline=64,firstnumber=50,
    float=*,label={lst-tex-ans},caption={The generated block in \fn{hw.ans.tex}}
]{hw.ans.tex}
\lstinputlisting[
    firstline=57,lastline=67,firstnumber=57,
    float=*,label={lst-tex-sol},caption={The generated block in \fn{hw.sol.tex}}
]{hw.sol.tex}

\subsection{Using guards to conditionally define environments}

The implementation of the environments \env{question}, \env{answer}, \env{hint},
and \env{solutions} have to be set up in the preambles of the generated \LaTeX{}
files (or in packages used by them).  But guards can be used in the preambles
too.  In this way, we can conditionally style the document.  

For instance, I prefer that the question text be upright in the questions file
and italicized in the answers/solutions file.  This is accomplished in
Listing~\ref{lst-dtx-amsthm}.  Line~34 is written to the answers and solutions
file, and overrides line~33.  The preamble of the questions file defines
\env{question} under the \verb|definition| theorem style, with bold header and
upright body font.  But in the answers and solutions file, the \verb|plain|
theorem style is in force, so \env{question} sets its body in italic.

\lstinputlisting[
    firstline=31,lastline=35,firstnumber=31,
    float=*,label={lst-dtx-amsthm},
    caption={%
        An excerpt of \fn{hw.dtx} showing conditional styling of the \env{question}
        environment%
    }
]{hw.dtx}

\subsection{Advanced tricks}

If you look in the full \fn{hw.dtx} file online, you'll see a few more automatic
variations with \docstrip:

\begin{itemize}
    \item The document title is set in \fn{hw.qns.tex}.  In \fn{hw.ans.tex}, the
    text \verb|Answers to | is prepended to the title (using the \cs{preto} command
    from the \pkg{etoolbox} package).  In \fn{hw.sol.tex}, the phrase
    \verb|Solutions to | is prepended.  In this way, the original title only needs
    to be put in one place.

    \item The document author is set differently in the different document
    variants. In \fn{hw.qns.tex} and \fn{hw.sol.tex}, the author is the
    professor.  In \fn{hw.ans.tex}, the author is set to a generic student name,
    and a \cs{LaTeXWarning} is used to remind the student to change the generic
    name to their own name.
\end{itemize}

\section{Comparing alternatives}

The problem of ``document options'' has been encountered by users before, for
instance on the StackExchange nextwork. 
\cite{TSE-docopts} 
\cite{TSE-makefile}
\cite{SO-texclargs}
Let's consider some of the alternatives in the context of the example use case
from Section~\ref{sec-exam}.

\subsection{Separate files}

In this strawman workflow, separate, nearly identical files are kept
side-by-side.  Any correction (to a problem, for instance) requires three files
be updated.  Further, copying a question to another assignment requires copying
from three different files to three different files.  This setup is ripe for
inconsistency and headache.

\subsection{Option setting in pure \texorpdfstring{\TeX}{TeX}}

In this workflow, certain \LaTeX{} booleans are defined and set, and code is
conditionally executed depending on those booleans.  The booleans can be set by
adding \TeX{} code on the command line as optional arguments to the executable
(e.g., \prog{pdflatex}).  Or, a shell file can be created which sets options,
then inputs a common master file.

When options are set within the document, or on the command line, the
\cs{jobname} is the same independent of the options set.  In our homework
example, this would mean the problems file and and solutions file would both end
up named \fn{hw.pdf}.  Not only does this the mean that the problem set and
solutions \acro{PDF}s cannot inhabit the same directory at the same time, one
can be mistaken for another.  Imagine the poor professor distributing what he
thought was the questions-only \acro{PDF}, only to realize that he had instead
shared all the solutions!

With shell files, the \cs{jobname} is different for each document variant,
avoiding the possibility of such a mistake.  But each bundle of documents
requires now creating several files: one master, and one shell file for each
variant.  A Makefile could automate creation of the shell files, but it's still
a separate file.

Also, in the context of a homework assignment, none of these methods allows
the distribution of an answer template.  If the solutions are in the master
\TeX{} file, and conditonally kept or discarded depending on options, they
remain in the source file no matter what.  An answer template with solutions
hidden in the source file defeats the purpose of a homework assignment.

\subsection{Symlinks}

In this workflow, a single \TeX{} file is created, and each variant document is
a symbolic link to the original file with a different file name.  The operating
system treats a symbolic link to a file as a different name for that file.
Editing one of these ``files'' effects the contents referenced by the file and
all of its symlinks.  But the \cs{jobname} is determined by the file name, so it
can be tested in order to conditionally execute certain code.

This avoids the colliding output file issue of command-line arguments, and is
more lightweight than shell files.  A new variant just requires a new symlink.
It has the same disadvantages of the master-plus-shell files workflow, too. Code
cannot be stripped out of the master file for a template, only gobbled and
discarded.  Also, symbolic links require a certain level of comfort with the
operating system in use.

The \docstrip{} method described here is operating system independent.  It is
secure in that each document variant gets a distinct \cs{jobname} and output
file name.  It is lightweight in that only one \docstrip{} file needs to be
created for every bundle of documents needed, and no programming other than \TeX
is necessary.

\subsection{Disadvantages of the \docstrip{} method}

One drawback of this method is that it requires an extra \TeX{} run.  First, the
\docstrip{} file is compiled, extracting the various \TeX{} files.  Then each of
them must be compiled.  A bit of programming can automate this process (as well
as decide when certain files don't need to be recompiled), as we'll describe in
the next section.  

Another, more uncomfortable drawback is that \TeX{} errors are only discovered
when the generated \TeX{} files are compiled.  The line numbers reported by
\TeX{} at the time of the error are different from the line numbers in the
\docstrip{} file.  So diagnosing errors needs to be done without navigating to
specific line numbers.  Rather, the token list before the error can be used for
a search string.

Finally, errors in the first run of \TeX{} (when \docstrip{} is extracting) can
arise from unbalanced guard modifiers, e.g., a \verb|%<*solutions>| line with no
closing \verb|%</solutions>|.  These are hard to isolate since \docstrip{} does
not log its processing with much context, and the error isn't discovered until
the end of the file is reached..  I have been able to find these through a
combination of retracing my steps, and selectively commenting out blocks.

\section{Automation}

To recap, this workflow requires editing a \docstrip{} file marked up as in
Section~\ref{sec-exam}, compiling that \docstrip{} file to generate separate
\TeX{} files, then compiling the desired \TeX{} file.  The first run can be done
in any \TeX engine, because the batch declaration header (and \fn{docstrip.tex})
is in core \TeX{}.  The second set of compilations require whatever engine your
destination documents require.

The \prog{latexmk} program \cite{latexmk} works like \prog{make} for \TeX{}
projects.  It examines a \TeX{} file to find dependencies, watces for warnings
about re-running \LaTeX, ad runs the necessary commands to get the entire
document stable.  \prog{latexmk} is written in perl, distributed with \TeX-live,
and actively maintained.

A unix one-line command that does both of these is:
\begin{lstlisting}[language=bash,numbers=none]
    tex foo.dtx && latexmk 
\end{lstlisting}
This processes \fn{foo.dtx} and, upon success of that command, runs
\prog{latexmk}.  Without file arguments, \prog{latexmk} looks for any \TeX{}
files in the current working directory and makes them.  Any command-line options
to \prog{latexmk} (e.g., \opt{-pdf} to make sure the \prog{pdftex} engine is
chosen, \opt{-pdfxe} to ensure \prog{xelatex}, and \opt{-pdflua} to ensure
\prog{lualatex}) will be passed when making each \TeX{} file.

The process can be further automated and integrated into various \TeX{} editors.
I have written a \fn{.latexmkrc} (configuration file for \prog{latexmk}) that
looks for \docstrip{} files in the argument list, and when found, parses their
log files for names of generated files to make automatically.  With this
configuration, \lstinline[basicstyle=\ttfamily]|latexmk foo.dtx| will take care
of all generated files in one fell swoop.

Any editor that can run a program can be configured to run \prog{latexmk}.  For
instance, I have written a \TeX{}Shop ``engine'' script that wraps around
\prog{latexmk} so configured.  I edit the \docstrip{} and press Command-T once.
I open one of the generated files in ``preview'' mode, which gives the PDF
window but not the \TeX{} window (we won't be editing the generated file
directly, so we don't need it.)  The preview window updates each time the
underlying file is changed.

I have also gotten this workflow to succeed in Visual Studio Code with the
\LaTeX{} Workshop \cite{vscode-latex-workshop}  These script files and
corresponding documentation are in the github repository.

\section{Conclusion}

I will continue to maintain and update the github repository referenced above.
If you would like to try this method, and find that additional documentation
would be useful, I will be happy to include it.

\bibliographystyle{tugboat}
\bibliography{dtxvar}

\makesignature
\end{document}