% This template is public domain.
\documentclass{ltugboat}

\usepackage{microtype}
\usepackage{graphicx}
\usepackage{ifpdf}
\ifpdf
\usepackage[breaklinks,hidelinks,pdfa]{hyperref}
\else
\usepackage{url}
\fi

\newcommand{\cls}[1]{\texttt{#1}}
\newcommand{\pkg}[1]{\texttt{#1}}
\newcommand{\docstrip}{\textsf{DocStrip}}

%%% Start of metadata %%%
\title{Using \docstrip{} for Multiple Document Variants}

% repeat info for each author.
\author{Matthew Leingang}
\address{New York University \\ 251 Mercer Street \\ New York, NY 10012\\ USA}
\netaddress{leingang@nyu.edu}
\personalURL{https://cims.nyu.edu/~leingang/}

%%% End of metadata %%%

\begin{document}

\maketitle

\begin{abstract}
    I describe a method of keeping multiple variants of the same document
    within the same file using \docstrip.
\end{abstract}

\section{Introduction}

As a college professor, there are several times when I need to keep teaching
materials in several different forms.  For example:

\begin{itemize}
    \item A single class day's lesson might consist of lecture slides in the
    \cls{beamer} class, a handout of the same slides printed 2–3 on a page for
    student notes, my own lecture notes in \cls{beamer} with the
    \pkg{beamerarticle} package loaded, a worksheet for distribution, and
    solutions to the worksheet.

    \item A single week's homework assignment might consist of problem
    statements, with hints and reading notes, a \LaTeX{} template for students
    to fill in with their own answers, and solutions with comments to be
    published after the assignment has been graded.
\end{itemize}

Over the years I have developed a workflow for writing these bundles of
documents in the same file, using the \LaTeX{} \docstrip{} utility.  In this
article I will introduce the reader to \docstrip{} and explain how I use it.

\section{The \docstrip{} utility}

\section{Example: A problem set with answer template and solutions}

\section{Comparing alternatives}

\section{Automation}

\section{Conclusion}

\bibliographystyle{tugboat}% or plain if you don't have tugboat.bst
\nocite{book-minimal}      % just making the bibliography non-empty
\bibliography{xampl}       % xampl.bib comes with BibTeX

\makesignature
\end{document}